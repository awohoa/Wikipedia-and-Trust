% Introduction
% Structure: Problem/Motivation -> Background -> Research Questions -> Contributions -> Paper Outline

% % Motivating example or opening
% [Start with a compelling example or statement about Wikipedia governance]

% % The problem/question
% [What is the specific problem or question you're investigating?]

% % Why it matters
% [Why is this important? Who cares about this problem?]
Wikipedia serves as a cornerstone of online information sharing, allowing users worldwide to collaboratively create and edit content. Despite its popularity, 
questions persist about how trustworthy its articles are, given that anyone can contribute edits. Traditional views associate trustworthiness with professional oversight, 
but Wikipedia’s strength lies in its community-driven revision system. Frequent edits and transparent revision histories may provide a unique measure of accountability and 
reliability.

This study examines whether the number of revisions an article receives can serve as a proxy for its trustworthiness. If more revisions correlate with well-sourced, 
balanced, and accurate articles, then revision count could be used as a simple quantitative indicator of reliability. Conversely, if frequent revisions occur mostly on 
controversial or low-quality pages, it may indicate that editing activity alone is not a sufficient measure of trust

% Research questions
This paper investigates the following research questions:
\begin{enumerate}
    \item Does the number of revisions to a Wikipedia article correlate with its perceived or measured trustworthiness?
    \item Do highly revised articles contain more citations or show more balanced sentiment than less revised ones?
    \item How might revision patterns vary across article categories or topics?
\end{enumerate}




% Contributions
The main contributions of this work are:
\begin{itemize}
    \item First contribution - e.g., novel analysis of X
    \item Second contribution - e.g., findings about Y
    \item Third contribution - e.g., methodology for Z
\end{itemize}

% Paper outline
The following sections describe our approach to analyzing Wikipedia’s revision data using the Wikipedia API. We extracted metadata such as revision counts, article lengths,
 and the number of cited sources to explore how these factors relate to perceived trustworthiness. To investigate these relationships, we collected a randomized sample of 
 Wikipedia articles and analyzed key metrics including revision frequency, contributor diversity, and sentiment balance. The next section outlines our data collection 
 process and the tools used to conduct this analysis.
 
% The rest of this paper is organized as follows. 
Section~\ref{sec:related} reviews related work on Wikipedia governance. 
Section~\ref{sec:methodology} describes our data and methods.
Section~\ref{sec:results} presents our findings.
Section~\ref{sec:discussion} discusses implications and limitations.
Section~\ref{sec:conclusion} concludes and suggests future work.

