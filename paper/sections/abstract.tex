% Abstract: 150-250 words summarizing your entire paper
% Include: problem, methods, key findings, contribution

Wikipedia is one of the world’s most widely used open-source knowledge platforms, but its reliability is often questioned due to the fact that anyone can edit its content. 
One potential indicator of an article’s trustworthiness is the number of revisions it has undergone. Frequent updates and changes on articles may reflect active community 
oversight and continuous quality improvement. This study investigates whether a higher number of revisions correlates with increased article reliability. Using the Wikipedia
 API, we collect data on random Wikipedia articles, including their revision histories, citation counts, article lengths, and contributor information. We then perform a 
 basic sentiment analysis on article text to explore whether tone or neutrality relates to revision frequency. By analyzing how revision activity connects to article 
 characteristics associated with trust, we aim to understand whether collaborative editing improves information quality. Our results will contribute to ongoing discussions
  about open-source knowledge governance and may help identify measurable indicators of reliability within large, user-generated platforms like Wikipedia.

