% Methodology
% Be specific enough that someone could reproduce your work

\label{sec:methodology}

% Overview
Data was collected using MediaWiki API, which provides access to articles and metadata from Wikipedia. Wikipedia data is publicly accessible and no private user data is 
collected.To ensure a diverse and unbiased sample of articles are selected, the random generator parameter is used. The following data is retrieved: 

\begin{itemize}
    \item Extracted text content (plain text summary)
    \item Article Title
    \item Date of last revision
    \item Number of revision
    \item Number of contributors
    \item Article text length
    \item Number of external links
\end{itemize}


\subsection{Data Collection}

\begin{itemize}
    \item Tools including MediaWiki API and Python’s requests library were used. 
    \item API Parameters used:
    
\begin{itemize}
    \item “action” : “query”
\begin{itemize}
    \item To retrieve article metadata
\end{itemize}
    \item “format” : “json”
\begin{itemize}
    \item Ensures data parsing
\end{itemize}
    \item "generator" : "random"
\begin{itemize}
    \item Select random articles
\end{itemize}
    \item "prop" : "revisions|extracts|contributors|extlinks"
\begin{itemize}
    \item Obtains information about revisions, summaries, contributors, and sources
\end{itemize}
    \item "ellimit" : "max"
\begin{itemize}
    \item Retrieves all external links
\end{itemize}
    \item "rvlimit" : 1
\begin{itemize}
    \item Limits revision data to 1 to ensure efficiency
\end{itemize}
\end{itemize}
\end{itemize}


% Example of how to cite a dataset or tool:
% We collected data from the Wikipedia API \cite{wikipedia-api} covering...

% \subsection{Data Processing}

% [Describe how you cleaned, filtered, or transformed your data]

% % Example of code reference:
% % Our data processing pipeline is available in our GitHub repository.\footnote{\url{https://github.com/yourusername/yourrepo}}

% \subsection{Analysis Methods}

% [Describe your analytical approach. What techniques did you use? Why?]

% % If you have equations, you can include them:
% % \begin{equation}
% % your\_formula = here
% % \end{equation}

% \subsection{Ethical Considerations}
% Optional: Include this if your project involves human subjects data, user behavior analysis, or if your institution requires ethics discussion. For basic Wikipedia article analysis, this may not be necessary - just ensure proper citation in your Data Collection section.

% Example: All data consists of publicly available Wikipedia content accessed in compliance with \href{https://foundation.wikimedia.org/wiki/Policy:Terms_of_Use}{Wikimedia's Terms of Use}.

